\documentclass[11pt,preprint, authoryear]{elsarticle}

\usepackage{lmodern}
%%%% My spacing
\usepackage{setspace}
\setstretch{1.2}
\DeclareMathSizes{12}{14}{10}{10}

% Wrap around which gives all figures included the [H] command, or places it "here". This can be tedious to code in Rmarkdown.
\usepackage{float}
\let\origfigure\figure
\let\endorigfigure\endfigure
\renewenvironment{figure}[1][2] {
    \expandafter\origfigure\expandafter[H]
} {
    \endorigfigure
}

\let\origtable\table
\let\endorigtable\endtable
\renewenvironment{table}[1][2] {
    \expandafter\origtable\expandafter[H]
} {
    \endorigtable
}


\usepackage{ifxetex,ifluatex}
\usepackage{fixltx2e} % provides \textsubscript
\ifnum 0\ifxetex 1\fi\ifluatex 1\fi=0 % if pdftex
  \usepackage[T1]{fontenc}
  \usepackage[utf8]{inputenc}
\else % if luatex or xelatex
  \ifxetex
    \usepackage{mathspec}
    \usepackage{xltxtra,xunicode}
  \else
    \usepackage{fontspec}
  \fi
  \defaultfontfeatures{Mapping=tex-text,Scale=MatchLowercase}
  \newcommand{\euro}{€}
\fi

\usepackage{amssymb, amsmath, amsthm, amsfonts}

\def\bibsection{\section*{References}} %%% Make "References" appear before bibliography


\usepackage[round]{natbib}

\usepackage{longtable}
\usepackage[margin=2.3cm,bottom=2cm,top=2.5cm, includefoot]{geometry}
\usepackage{fancyhdr}
\usepackage[bottom, hang, flushmargin]{footmisc}
\usepackage{graphicx}
\numberwithin{equation}{section}
\numberwithin{figure}{section}
\numberwithin{table}{section}
\setlength{\parindent}{0cm}
\setlength{\parskip}{1.3ex plus 0.5ex minus 0.3ex}
\usepackage{textcomp}
\renewcommand{\headrulewidth}{0.2pt}
\renewcommand{\footrulewidth}{0.3pt}

\usepackage{array}
\newcolumntype{x}[1]{>{\centering\arraybackslash\hspace{0pt}}p{#1}}

%%%%  Remove the "preprint submitted to" part. Don't worry about this either, it just looks better without it:
\makeatletter
\def\ps@pprintTitle{%
  \let\@oddhead\@empty
  \let\@evenhead\@empty
  \let\@oddfoot\@empty
  \let\@evenfoot\@oddfoot
}
\makeatother

 \def\tightlist{} % This allows for subbullets!

\usepackage{hyperref}
\hypersetup{breaklinks=true,
            bookmarks=true,
            colorlinks=true,
            citecolor=blue,
            urlcolor=blue,
            linkcolor=blue,
            pdfborder={0 0 0}}


% The following packages allow huxtable to work:
\usepackage{siunitx}
\usepackage{multirow}
\usepackage{hhline}
\usepackage{calc}
\usepackage{tabularx}
\usepackage{booktabs}
\usepackage{caption}


\newenvironment{columns}[1][]{}{}

\newenvironment{column}[1]{\begin{minipage}{#1}\ignorespaces}{%
\end{minipage}
\ifhmode\unskip\fi
\aftergroup\useignorespacesandallpars}

\def\useignorespacesandallpars#1\ignorespaces\fi{%
#1\fi\ignorespacesandallpars}

\makeatletter
\def\ignorespacesandallpars{%
  \@ifnextchar\par
    {\expandafter\ignorespacesandallpars\@gobble}%
    {}%
}
\makeatother

\newlength{\cslhangindent}
\setlength{\cslhangindent}{1.5em}
\newenvironment{CSLReferences}%
  {\setlength{\parindent}{0pt}%
  \everypar{\setlength{\hangindent}{\cslhangindent}}\ignorespaces}%
  {\par}


\urlstyle{same}  % don't use monospace font for urls
\setlength{\parindent}{0pt}
\setlength{\parskip}{6pt plus 2pt minus 1pt}
\setlength{\emergencystretch}{3em}  % prevent overfull lines
\setcounter{secnumdepth}{5}

%%% Use protect on footnotes to avoid problems with footnotes in titles
\let\rmarkdownfootnote\footnote%
\def\footnote{\protect\rmarkdownfootnote}
\IfFileExists{upquote.sty}{\usepackage{upquote}}{}

%%% Include extra packages specified by user

%%% Hard setting column skips for reports - this ensures greater consistency and control over the length settings in the document.
%% page layout
%% paragraphs
\setlength{\baselineskip}{12pt plus 0pt minus 0pt}
\setlength{\parskip}{12pt plus 0pt minus 0pt}
\setlength{\parindent}{0pt plus 0pt minus 0pt}
%% floats
\setlength{\floatsep}{12pt plus 0 pt minus 0pt}
\setlength{\textfloatsep}{20pt plus 0pt minus 0pt}
\setlength{\intextsep}{14pt plus 0pt minus 0pt}
\setlength{\dbltextfloatsep}{20pt plus 0pt minus 0pt}
\setlength{\dblfloatsep}{14pt plus 0pt minus 0pt}
%% maths
\setlength{\abovedisplayskip}{12pt plus 0pt minus 0pt}
\setlength{\belowdisplayskip}{12pt plus 0pt minus 0pt}
%% lists
\setlength{\topsep}{10pt plus 0pt minus 0pt}
\setlength{\partopsep}{3pt plus 0pt minus 0pt}
\setlength{\itemsep}{5pt plus 0pt minus 0pt}
\setlength{\labelsep}{8mm plus 0mm minus 0mm}
\setlength{\parsep}{\the\parskip}
\setlength{\listparindent}{\the\parindent}
%% verbatim
\setlength{\fboxsep}{5pt plus 0pt minus 0pt}



\begin{document}



\begin{frontmatter}  %

\title{Utilising Random Forest Algorithms to Classify Those Most Likely to Lose
Their Main Source of Income due to Lockdown- Evidence From NIDS-CRAM
Wave 1}

% Set to FALSE if wanting to remove title (for submission)




\author[Add1]{Johannes Coetsee - 19491050}
\ead{19491050@sun.ac.za - https://github.com/Coetsee}





\address[Add1]{Stellenbosch University}



\vspace{1cm}

\vspace{0.5cm}
\end{frontmatter}



%________________________
% Header and Footers
%%%%%%%%%%%%%%%%%%%%%%%%%%%%%%%%%
\pagestyle{fancy}
\chead{}
\rhead{July 2021 - Data Science 871}
\lfoot{}
\rfoot{\footnotesize Page \thepage}
\lhead{}
%\rfoot{\footnotesize Page \thepage } % "e.g. Page 2"
\cfoot{}

%\setlength\headheight{30pt}
%%%%%%%%%%%%%%%%%%%%%%%%%%%%%%%%%
%________________________

\headsep 35pt % So that header does not go over title




\hypertarget{introduction}{%
\section{\texorpdfstring{Introduction
\label{Introduction}}{Introduction }}\label{introduction}}

The purpose of this paper is to report on the implementation of a Random
Forest (RF) algorithm for a classification-type problem, namely, to
classify which individuals and households were more likely to lose their
main source of income due to the coronavirus and subsequent lockdown in
South Africa in March and April 2020.\footnote{The template for this
  report is based on that provided by Katzke
  (\protect\hyperlink{ref-Texevier}{2017}).} RF is well-suited for
classification-type problems, as

\hypertarget{data}{%
\section{\texorpdfstring{Data \label{Data}}{Data }}\label{data}}

This study utilises the first wave of the National Income Dynamics Study
- Coronavirus Rapid Mobile Survey 2020 (NIDS-CRAM) dataset, a
longitudinal telephonic household survey conducted by the Southern
Africa Labour and Development Research Unit (SALDRU) in April and May
2020. NIDS-CRAM investigates the various social and economic effects of
the national lockdown implemented in March 2020, and more broadly, the
consequences of the global pandemic on the South African population.

In total, the dataset consists of 21 features, which is reported in
Table \ref{table} below, with 7073 observations for each feature. The
main variable of interest is `Income.Change' - a binary variable where a
value of 1 indicates that the household has lost their main source of
income, whilst 2 indicates that it has not. The question asked to
respondents reads as : ``Has your household lost its main source of
income since the lockdown started on 27th March?''

\hypertarget{missing-values-and-transformations}{%
\subsection*{Missing Values and
Transformations}\label{missing-values-and-transformations}}
\addcontentsline{toc}{subsection}{Missing Values and Transformations}

In order to avoid losing information, this study imputes missing values
for the entire dataset of 21 features. Before imputation, however, NAs
for the `Employment.Type' feature are replaced by 0's to indicate
`unemployed', as this survey question was only asked to those who were
employed. Those who refused to respond or did not know their main form
of work, were indicated as missing and therefore imputed. Similarly,
system NAs for the `Tertiary' feature - a dummy variable indicating
whether an individual completed some form of tertiary education - was
replaced by 0, or `no', as this question was only asked to those who
were eligible. Although not perfect solutions, these are fair
assumptions to make in order to include these potentially important
variables. Additionally, the feature indicating in which Municipal
Demarcations Board District Council the household is situated was
transformed into a matrix of binary variables so as to accommodate the
necessary structure needed for imputation.\footnote{This is due to the
  fact that only 53 levels are allowed for factored variables using the
  \emph{missForest} and \emph{randomForest} packages, whereas the
  District Council variable consists of 54.}

The method of imputation used in this paper draws on a random forest
algorithm to impute missing values trained on the matrix of observed
values in the data. This can be done using the package \emph{missForest}
in R, following a two-step procedure. First, missing values are
pre-imputed using simple median replacement - where the missing value is
replaced with the median value computed on the rest of the observed data
for each continuous feature. For categorical variables, missing values
are replaced by the most frequently occurring non-missing
value.\footnote{This process is also called Strawman imputation.}
Second, a forest is grown using multivariate splitting, where the
splitting rule is averaged only over non-missing values. Data is then
imputed by regressing each feature on all other features, thereafter
predicting missing values using the fitted forest. This process is
iterated in order to update the initial median-replaced values until the
stopping criterion - in our case, when the difference between the
previous iteration and new iteration have become larger once for each
data type - is met (Tang \& Ishwaran,
\protect\hyperlink{ref-tang2017random}{2017}).

The usage of this algorithm is necessitated by the nature of the data,
where features are of three different data types, namely, categorical,
numeric and continuous. Stekhoven \& Bühlmann
(\protect\hyperlink{ref-stekhoven2012missforest}{2012}) show that this
iterative imputation procedure outperforms many other widely-used
implementation methods such as, for instance, K-Nearest Neighbours (KNN)
imputation and Multivariate Imputation by Chained Equations (MICE),
especially within mixed-type data contexts. Furthermore, it inherits all
the positive attributes attributed to random forests itself, such as
being robust to noisy data due to inherent feature selection, as well as
being simple to implement. However, it is computationally intensive and
crucially also relies on the assumption that missing data are Missing At
Random (MAR). If not MAR, there is possibility of introduced selection
bias. This is deemed a permissible admission due to the relatively low
number of missing values in the dataset as a whole.

\hypertarget{methodology}{%
\section{\texorpdfstring{Methodology
\label{Meth}}{Methodology }}\label{methodology}}

\hypertarget{cloud-computing}{%
\subsection*{Cloud Computing}\label{cloud-computing}}
\addcontentsline{toc}{subsection}{Cloud Computing}

\hypertarget{sql}{%
\subsection*{SQL}\label{sql}}
\addcontentsline{toc}{subsection}{SQL}

\hypertarget{the-random-forest-algorithm}{%
\subsection*{The Random Forest
Algorithm}\label{the-random-forest-algorithm}}
\addcontentsline{toc}{subsection}{The Random Forest Algorithm}

\hypertarget{gbm}{%
\subsection*{GBM}\label{gbm}}
\addcontentsline{toc}{subsection}{GBM}

\hypertarget{results}{%
\section{Results}\label{results}}

\hypertarget{model-1-random-forest}{%
\subsection{Model 1: Random Forest}\label{model-1-random-forest}}

\hypertarget{prediction-and-confusion-matrix-train-vs-test-data}{%
\subsubsection*{prediction and confusion matrix, train vs test
data}\label{prediction-and-confusion-matrix-train-vs-test-data}}
\addcontentsline{toc}{subsubsection}{prediction and confusion matrix,
train vs test data}

\hypertarget{error-rate-and-bootstrap-samples}{%
\subsubsection*{error rate and bootstrap
samples}\label{error-rate-and-bootstrap-samples}}
\addcontentsline{toc}{subsubsection}{error rate and bootstrap samples}

\hypertarget{number-of-nodes}{%
\subsubsection*{number of nodes}\label{number-of-nodes}}
\addcontentsline{toc}{subsubsection}{number of nodes}

\hypertarget{hyperparameter-tuning}{%
\subsubsection*{hyperparameter tuning}\label{hyperparameter-tuning}}
\addcontentsline{toc}{subsubsection}{hyperparameter tuning}

\hypertarget{variable-importance}{%
\subsubsection*{variable importance}\label{variable-importance}}
\addcontentsline{toc}{subsubsection}{variable importance}

\hypertarget{partial-dependence-plot}{%
\subsubsection*{partial dependence plot}\label{partial-dependence-plot}}
\addcontentsline{toc}{subsubsection}{partial dependence plot}

\hypertarget{model-2-gbm-random-forest}{%
\subsection{Model 2: GBM Random
Forest}\label{model-2-gbm-random-forest}}

gri

\hypertarget{conclusion}{%
\section{Conclusion}\label{conclusion}}

\newpage

\hypertarget{references}{%
\section*{References}\label{references}}
\addcontentsline{toc}{section}{References}

\hypertarget{refs}{}
\leavevmode\hypertarget{ref-Texevier}{}%
Katzke, N.F. 2017. \emph{Texevier: Package to create elsevier templates
for rmarkdown}. ed. Stellenbosch, South Africa: Bureau for Economic
Research.

\leavevmode\hypertarget{ref-stekhoven2012missforest}{}%
Stekhoven, D.J. \& Bühlmann, P. 2012. MissForest---non-parametric
missing value imputation for mixed-type data. \emph{Bioinformatics}.
28(1):112--118.

\leavevmode\hypertarget{ref-tang2017random}{}%
Tang, F. \& Ishwaran, H. 2017. Random forest missing data algorithms.
\emph{Statistical Analysis and Data Mining: The ASA Data Science
Journal}. 10(6):363--377.

\hypertarget{appendix}{%
\section*{Appendix}\label{appendix}}
\addcontentsline{toc}{section}{Appendix}

\hypertarget{appendix-a}{%
\subsection*{Appendix A}\label{appendix-a}}
\addcontentsline{toc}{subsection}{Appendix A}

\bibliography{Tex/ref}





\end{document}
